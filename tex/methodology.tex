% A workaround to allow relative paths in included subfiles
% that are to be compiled separately
% See https://tex.stackexchange.com/questions/153312/subfiles-inside-a-subfile-using-relative-paths
\providecommand{\main}{..}
\documentclass[\main/thesis.tex]{subfiles}

\begin{document}

\chapter{Methodology}\label{chap:methodology}

\section{Obtaining Models}\label{sec:ObtainingModels}
To begin with, we obtained models from two sources. The first model was the COMET-DISTIL model
which was downloaded from the original author's GitHub page\cite{west_symbolic_2021}. The second model 
was the naive GPT-2 XL model. This model was obtained from the Hugging Face model hub by using their API.\@

Once the two models and their tokenizers were downloaded, we set up a series of experiments with 
different datasets. All of this was done on the University of Alberta Lambda Compute Instance via an SSH tunnel. 

\section{Preparing Datasets}\label{sec:PrepareDataset}
To evaluate and compare the performance of our two models, it was crucial to select a benchmark 
that provides a comprehensive assessment across various linguistic tasks. 
After careful consideration, the General Language Understanding Evaluation (GLUE)\cite{wang_glue_2019} benchmark was chosen for the following reasons:

\begin{enumerate}
    \item \textbf{Comprehensive Assessment}: GLUE is an aggregation of multiple data sources and 
    tasks that span a broad range of NLP challenges. This includes tasks such as sentiment analysis, 
    question-answering, and textual entailment, among others. Using GLUE allows for a holistic 
    evaluation of the LLM's capabilities.

    \item \textbf{Standardization}: Since its inception, GLUE has been widely adopted in the 
    NLP community as a standard for evaluating language understanding capabilities. 
    Comparing our models against this standard ensures that our results are in line with 
    the broader research community's expectations and methodologies.

    \item \textbf{Reproducibility and Fairness}: Given the well-defined evaluation metrics and 
    standardized tasks, using GLUE ensures that our comparisons between the two LLM versions are 
    fair and can be reliably reproduced by other researchers.

    \item \textbf{Model Generalization}: By evaluating on GLUE, we can ascertain how well our 
    LLM versions generalize across diverse tasks. This is crucial for understanding potential 
    areas of improvement and the model's applicability in real-world scenarios.

    \item \textbf{Community Recognition}: The results on GLUE are easily interpretable and 
    recognizable by those familiar with NLP research. This facilitates clearer communication of 
    our model's strengths and areas for improvement to peers and stakeholders.

\end{enumerate}

Given these advantages, the GLUE benchmark provided an objective and widely-accepted means 
to compare and contrast the performance of our two LLM versions, ensuring both rigorous 
evaluation and clear communication of our findings.

We were able to download the dataset directly from the Hugging Face website. There is an accessible
API that allows direct use of the datasets from the Hugging Face library within the model 
training code which we were able to capitalize on after. 

\section{Training Models}\label{sec:ModelTraining}

We tried several approaches including writing custom code to fine tune the models for 
the specific tasks in the GLUE dataset. We eventually discovered that the Transformers library 
comes with a handy run\_glue.py script that can be found in their GitHub repository.

In order to get everything working correctly, we had to first install the Transformers library 
from source. Then we had to install the DeepSpeed library and try several different configurations 
to get the models to fit in memory and be able to train in a reasonable amount of time. 


We also modified the code for the run\_glue.py script as we noticed some issues while trying to collect
some of the results for the GLUE benchmark dataset. 


In order to conduct our study, we fine-tuned both models on the training subset for each of the GLUE 
benchmark sets. Then we evaluated them against the Evaluation set. Since the evaluation and the 
training sets are predefined, we opted to run the experiments 5 times with a different random variable set 
for the model training. 

We then collected the predictions for the task from both models where we noticed 
large differences. This was done to understand if there were particular scenarios/patterns where the 
models did differently. 

\section{Conducting Tests and Collating Results}

The training and testing of the two models took a little over 48 days to complete. Computationally, 
LLMs are hard to work with for this reason. We were limited in the variety of benchmarks we could run
as even fine-tuning and evaluating these models without significantly more computing resources is very 
time-consuming. 

We tracked these experiment and collected these results by using the Weights and Biases framework. The 
W\&B report can be found via this link: 

https://api.wandb.ai/report/ece-910/832fd6di

\end{document}